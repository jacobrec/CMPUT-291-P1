The application has 3 separate visible screen. Each screen has several
available commands. They will be described in this section. Navigation
throughout the application is to be done using the keyboard, typing in the full
command name, or the number to navigate between screens. Then typing in the
required information when prompted to do so. Please take care to insure all
input is correct. The application assumes the user is to be trusted, however,
unwanted modifications could occur in the database when used by someone
careless. At any point in the application, the user can hit Control-C to exit
the program, and Control-D to exit the current command.

The following subsections describe in more detail, the role and usage of each
of the accessible screens of the application. An overview of the screens is found in figure~\ref{fig:use_flow}

\begin{figure}
    \centering
    \includegraphics[width=\textwidth]{use_flow_dot.png}
    \caption{Screens and commands, and how they are accessed}\label{fig:use_flow}
\end{figure}

\subsection{Login Screen}

From this screen, the user is presented with 2 options. By pressing 1, or
typing out the word \verb|login| the user will be guided through the process of
logging into the application. The user will be prompted for their username,
then for their password. For the security of the user, unlike the username,
when typing the password, the password will not be displayed. Don't panic
though, they are still being picked up by the application.

Also on the screen is the other option, exit. By inputting the command 2, or
the full command name \verb|exit| the user will be able to safely close out of
the application. This is the only recommended way to exit the application.
Command line savvy users may try certain control presses to stop the program.
While these may work, they are not officially supported, and the use of such
will void the warranty of the user.

\subsection{Registry Agent Screen}
This screen has 7 options for the user to pick from. The options are
\verb|register a birth|, \verb|register a marriage|,
\verb|renew vehicle registration|, \verb|process bill of sale|,
\verb|process a payment|, \verb|get driver abstract|, and \verb|signout|.

Register a birth will request the following information in order: first name of
the newborn, last name of the newborn, the gender of the newborn, the birth
date, the birth place, the mother's first name, the mother's last name, the
father's first name, and the father's last name.  None of this information may
be excluded, and all inputs are case insensitive.  The comination of first and
last names for the child cannot already be taken.  The gender must be either
"M" for male, "F" for female, or "O" for other.  The birth date must be in
valid date format.  If the mother or father are not already registered in the
persons database, the user will be requested to add additional information for
them, including their birth date, birth place, address, and phone number.  Any
of these values may be empty.  The birth date must be in valid date format if
supplied, and the phone number must be in valid phone number format if supplied.
Once the parents have been added, the newborn will be created in the persons
table, using the same address and phone number as the mother.  Finally, the
registration for the birth is created.

Register a marriage requests the first and last names for both partners.  The
inputs are case insensitive.  If either of the partners do not exist in the
persons table the user will be prompted to fill in their details.  This includes
their birth date, birth place, address, and phone number.  Any of these values
may be empty.  The birth date must be in valid date format, and the phone number
in valid phone number format if they are supplied.  Once the partners are
confirmed to exist in the persons database, the marriage will be registered.

Renew a vehicle registration first asks the user for a valid registration
number for a vehicle. Once the user has entered a valid registration number the
registration will be renewed. If the registration number is invalid, it will
display that it could not find the registration and return you to the options.
On a successful renewal the new expiry date is chosen with a complex algorithm,
where if it is expired, it will be set to one year from todays date, and if it
is not expired, it will be set to one year from the current expiry date.

Process bill of sale asks the user for the vin number of the vehicle to be sold
and the name of the current owner of the vehicle. If the vin is invalid or the
given owner is not the most recent, the user will be informed and the process
will abort. Otherwise, the name for the new owner and the ew plate number will
be prompted. If the new owner does not exist in the database, the sale will
abort, otherwise the old registration will have it's expiry set to today and
a new registration will be generated.

Process a payment asks for a valid ticket number. If is does not exist, the user
will be informed and the command will exit. If a payment for the selected ticket
has already been processed today, then user will be in formed and the process will
abort. Otherwise, the ticket information will be displayed and the user will be
asked to confirm if they want to complete the payment. If so, the amount owed will
be displayed and the user can enter a number. If the number is greater that what is
owed, the process will exit. Otherwise, the payment will be entered.

Get driver abstract.

Signout signs out the currently signed in user, bringing them back to the main
menu where they can sign in again with the same or a different account, or exit
the program if they so wish.


\subsection{Traffic Officer Screen}

From this screen, the user is presented with 3 options. The options are
\verb|issue a ticket|, \verb|find a car owner|, and \verb|signout|.

Issue a ticket will first prompt the user to enter a vehicle registration
number. Once they enter a valid registration number, a brief description of the
registration will be shown. The user will then be prompted to decide if they
want to issue a ticket to the displayed registration. If the user decides not
to, the operation will be canceled, and the user will be returned to the
traffic officer screen. If the user decides to issue a ticket, then they will
be prompted to enter information about the date, amount, and reason for the
ticket.

Find a car owner prompts the user to describe the make, model, color, year, and
plate. Any, or all of these options can be left blank to be treated as a
wildcard option. For the convenience of the user, the application will not look
for exact matches, but will instead look for containing values. For example, the
user was searching for a Ford F150, they could type 15, and any models
containing the string 15 will be returned. If in the event there are many
matches to a particular search, the amount of data will be truncated, and the
user will be prompted to select one of the available rows, and that row will be
redisplayed with additional data.

Finally, signout can be used to return to the login screen.


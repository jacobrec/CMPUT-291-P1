Testing evolved through several phases as the project was developed.  Initially, individual rows were added to the database as needed, to test queries as they were being implemented.  This was effective when creating simpler queries such as the one used for the login procedure, but the scope of the test data was quite limited using this approach.  A more complete dataset was repurposed from Assignment #2 so that less time would be spent creating test cases from scratch.  However, testing still required the full interactive process - a slow and cumbersome affair, especially when errors occur near the end of a command. A script was developed to simulate a basic form of regression testing, running the program for a series of input keystrokes to make sure the output matched the expected text.  This allowed us to see if our changes impacted other parts of the code unintentionally.  Furthermore, testing could be done quicker and more thoroughly for each query, and be better documented.

User input errors were handled in code from the start, prompting the user to retype the input if the format didn't match what was required.  As a result, not much time was spent testing this.  A few bugs were discovered for inputs which violated primary key constraints, such as making multiple payments on the same date.  The program specification was adhered to as closely as possible, with clarifications from the class forum being incorporated at all stages of development to ensure that the expectations were met.  Case-insensitivity was tested thoroughly.

The application has 3 separate visible screen. Each screen has several
available commands. They will be described in this section. Navigation
throughout the application is to be done using the keyboard, typing in the full
command name, or the number to navigate between screens. Then typing in the
required information when prompted to do so. Please take care to insure all
input is correct. The application assumes the user is to be trusted, however,
unwanted modifications could occur in the database when used by someone
careless.

The following subsections describe in more detail, the role and usage of each
of the accessible screens of the application. An overview of the screens is found in figure~\ref{fig:use_flow}

\begin{figure}
    \centering
    \includegraphics[width=\textwidth]{use_flow_dot.png}
    \caption{Screens and commands, and how they are accessed}\label{fig:use_flow}
\end{figure}

\subsection{Login Screen}

From this screen, the user is presented with 2 options. By pressing 1, or
typing out the word \verb|login| the user will be guided through the process of
logging into the application. The user will be prompted for their username,
then for their password. For the security of the user, unlike the username,
when typing the password, the keystrokes will not be echoed back to the shell.
Don't panic though, they are still being picked up by the application.

Also on the screen is the other option, exit. By inputting the command 2, or
the full command name \verb|exit| the user will be able to safely close out of
the application. This is the only recommended way to exit the application.
Command line savvy users may try certain control presses to stop the program.
While these may work, they are not officially supported, and the use of such
will void the warranty of the user.

\subsection{Registry Agent Screen}


\subsection{Traffic Officer Screen}

From this screen, the user is presented with 3 options. The options are
\verb|issue a ticket|, \verb|find a car owner|, and \verb|signout|.

Issue a ticket prompts the user to describe the make, model, color, year, and plate.

A general overview of your system with a small user guide\\
The general overview of the system gives a high level introduction and may include a diagram showing the flow of data between different components; this can be useful for both users and developers of your application.
